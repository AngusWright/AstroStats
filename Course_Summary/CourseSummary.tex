% MNRAS Manuscript Template
% Written by A.H.Wright 2016
\documentclass[useams,useamsmath,usenatbib]{mnras}
\pdfminorversion=5
\usepackage{amsmath}
\usepackage{setspace}
\usepackage{pdfpages}
\graphicspath{{./images/}}

\def\ie{\,{\rm i.e.}\,}
\def\eg{\,{\rm e.g.}\,}
\def\um{{\rm $\mu$m}\,}
\def\sdss{{\sc sdss}}
\def\viking{{\sc viking}}
\def\hatlas{{\sc h-atlas}}
\def\wise{{\em WISE}}
\def\vista{{\sc vista}}
\def\galex{{\em GALEX}}
\def\bbf{}
%
% For Revisions: use \bbf to make revised text bold,
%                and uncomment the below. For final
%                submission, re-comment the line to
%                remove the bold-ness.
%\def\bbf{\bf}

% Arxiver Figure selection: Declare Figures to show on Arxiver
% ___Leave Commented out, just change the Figure Names___
% __Just use Figure Names, Path is determined internally__
%@arxiver{Figure1.png,Figure4.pdf,Figure5.eps}

\begin{document}

\onecolumn

\title[AstroStatistics]{Introduction to Statistics for Astronomers and Physicists}

%Authors {{{
\author[A.H. Wright]
{A.H.~Wright}
%\date{Received XXXX; Accepted XXXX}
%\pubyear{2016} \volume{000}
%\pagerange{\pageref{firstpage}--\pageref{lastpage}}
%}}}
\maketitle
\label{firstpage}
%\begin{abstract}%{{{
%%This paper is going to be pretty awesome \citep{CITATION}.
%\end{abstract}
%%}}}

\section{Course Summary}\label{sec: intro} %{{{
We provide an introduction to practical statistics for Astronomers and Physicists, who may have had little (or zero)
formal statistics education in their academic careers to date. The course is designed to develop a complex understanding
of statistical methods applicable to modern scientific inquiry, while generally using astronomical datasets for
exposition. The course modules are outlined below, and draws its content from a range of standard statistical texts. An
example reference list is given below.  Throughout the course we discuss standard statistical biases and mistakes, their
influence on the estimation of underlying models, and what students can do to fortify their science to common
statistical errors. The course will be run in weekly lectures (2 hours in duration) and without dedicated exercise
classes. Instead, in addition to lecture notes, the course will include hands-on examples to be completed during the
lectures, including computational examples covering all aspects of the course, written in both R and python. 

%\section{Course Modules}
The course is split into 4 core modules, each which cover a broad topic in statistics and data analysis. 
\paragraph*{Introduction to Probability: }
For all aspects of modern science, an understanding of probability is required. We cover a range of topics in
probability, from decision theory and the fundamentals of probability theory, to standard probabilistic distributions
and their origin.  From this module, students will gain an insight into different statistical distributions that govern
modern observational sciences, the interpretation of these distributions, and how one accurately summarises
distributions of data in an unbiased manner. 
\paragraph*{Data modelling: }
When working in empirical science, modelling and understanding datasets is paramount. In this module we start by
discussing the fundamentals of data modelling. We start by discussing theories of point and interval estimation, in the
context of summary statics (e.g. expectation values, confidence intervals), and estimation of data correlation and
covariance. Students will learn the fundamentals of data mining and analysis, in a way that is applicable to
all physical sciences.  
\paragraph*{Bayesian Statistics: }
Bayes theorem lead to a revolution in statistics, via the concepts of prior and posterior evidence. In modern astronomy
and physics, applications of bayesian statistics are widespread. We begin with a study of Bayes theorem, and the
fundamental differences between frequentest and Bayesian analysis. We explore applications of Bayesian statistics,
through well studied statistical problems (both within and outside of physics). 
\paragraph*{Parameter Simulation, Optimisation, \& Inference: }
We apply our understanding of Bayesian statistics to the common problems of parameter simulation, optimisation, and
inference. Students will learn the fundamentals of Monte Carlo Simulation, Markov Chain Monte Carlo (MCMC) analysis, hypothesis testing,
and quantifying goodness-of-fit. We discuss common errors in parameter inference, including standard physical and
astrophysical biases that corrupt statistical analyses. 

\paragraph*{Reference Texts}
\begin{enumerate}
\item[1.] Eadie, Drijard, James, Roos, \& Sadoulet, 1971, ``Statistical Methods in Experimental Physics'', ISBN:0-444-10117-9
\item[2.] Wall \& Jenkins, 2014, ``Practical Statistics for Astronomers'', ISBN:978-0-521-73249-9
\item[3.] Heumann \& Shalabh, 2016, ``Introduction to Statistics and Data Analysis'', ISBN: 978-3-319-46160-1
\item[4.] Marin \& Robert, 2014, ``Bayesian Essentials with R'', ISBN: 978-1-4614-8687-9
\item[5.] Schumacker \& Tomek, 2013, ``Understanding Statistics using R'', ISBN: 978-1-4614-6226-2
\end{enumerate}

%%Practical Statistics for Astronomy 
%Decisions 
%Probability 
%Statistics and expectations
%Correlation and association 
%Hypothesis testing 
%Data modelling and parameter estimation (basic) 
%Data modelling and parameter estimation (advanced) 
%Detection and Survey astronomy 
%Sequential data \& 1D statistics 
%Statistics of Large Scale Structures
%
%%Statistical Methods in Experimental Physics 
%Basic Concepts in Probability 
%Probability Distributions 
%Decision Theory 
%Theory of Estimators 
%Point Estimates in Practice
%Interval Estimation 
%Tests of Hypotheses
%Goodness-of-fit Tests
%
%%Introduction to Statistics and Data Analysis 
%PART I: DESCRIPTIVE STATISTICS 
%Introduction and Framework 
%Frequency Measures and Graphical Representation of Data 
%Measures of Central Tendency and Dispersion 
%Association of two variables 
%
%PART II: PROBABILITY CALCULUS 
%Combinatorics 
%Elements of Probability Theory 
%Random Variables 
%Probability Distributions 
%
%PART III: INDUCTIVE STATISTICS 
%Inference 
%Hypothesis Testing
%Linear Regression 
%





%}}}
%\bibliographystyle{mnras}
%\bibliography{library}
%\appendix%{{{
%\section{Appendix A}\label{sec: Appendix1} %{{{
%
%
%%}}}
\end{document}
